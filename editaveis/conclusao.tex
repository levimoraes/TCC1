\chapter[Conclusão]{Conclusão}
\label{sec:conclusão}
A escolha das métricas para análise, avaliação e acompanhamento de um software, é parte importante quando se fala de contratação de software. Contudo, a escolha das métricas é uma atividade muito subjetiva para que se possa definir contextos e situações específicas para um determinado grupo de métricas. A capacidade de avaliar as melhores métricas para determinados projetos, não está em um ferramenta mas sim na experiência, por isso gestores mais antigos tendem a usar métricas mais específicas e precisas para cada projeto. 

Este trabalhou teve como pergunta de pesquisa "Definido um conjunto de métricas, como criar um \textit{dashboard} que avalie a qualidade de software de um órgão público federal". Para responder a esta pergunta, tinha-se como objetivo geral a criação de um \textit{dashboard} que auxiliasse no acompanhamento da qualidade de código durante a fase de desenvolvimento. Pode-se dizer que este objetivo foi alcançado, devido ao alto índice no teste de usabilidade e alguns comentários positivos feitos por dois usuários entrevistados que trabalham com desenvolvimento dentro de Órgãos Públicos, contudo não se pode verificar a sua eficácia, pois não foi possível implantar a solução em um ambiente real.
Quanto aos objetivos específicos, todos foram realizados e validados também através do questionário de usabilidade quanto com a satisfação com que os entrevistados relatavam após a aplicação do questionário.

A Tabela \ref{tbl:table_status} apresenta os \textit{status} de completude do trabalho. Quase todas as atividades foram realizadas, apenas a atividades de Acompanhar Utilização do Software não foi realizada. A não realização da atividade se deve ao fato de que não foi possível a implantação da ferramenta em um ambiente real e por consequência não foi possível avaliar a eficácia da ferramenta.


\begin{table}[h!]
\centering
\caption{Tabela de Completude do Trabalho}
\label{tbl:table_status}
\begin{tabular}{lc}
\rowcolor[HTML]{9A0000} 
{\color[HTML]{FFFFFF} \textbf{Atividade}} & \multicolumn{1}{l}{\cellcolor[HTML]{9A0000}{\color[HTML]{FFFFFF} \textbf{Status}}} \\ \hline
Definir Tema                              & 100\%                                                                              \\ \hline
Validar Escopo                            & 100\%                                                                              \\ \hline
Elaborar Roteiro de Pesquisa              & 100\%                                                                              \\ \hline
Pesquisar Referência                      & 100\%                                                                              \\ \hline
Refinar Pesquisa                          & 100\%                                                                              \\ \hline
Catalogar Material                        & 100\%                                                                              \\ \hline
Documentar                                & 100\%                                                                               \\ \hline
Analisar Ambiente                         & 100\%                                                                               \\ \hline
Configurar Ambiente                       & 100\%                                                                                \\ \hline
Implementar Solução de Software           & 100\%                                                                                \\ \hline
Implantar Solução em Ambiente Simulado    & 100\%                                                                                \\ \hline
Acompanhar Utilização do Software         & 0\%                                                                                \\ \hline
Aplicar Questionário                      & 100\%                                                                                \\ \hline
\end{tabular}
\end{table}

\section{Trabalhos Futuros}
Uma possível trabalho derivado deste, seria um estudo de caso avaliando a implantação do software produzido em um órgão real, e acompanhar o quão relevante a ferramenta se mostrou no ambiente de desenvolvimento. Espera-se também que sejam definidos novos perfis de gestores, porém agora com base em gestores reais, para que se tenha um conjunto maior de perfis. Outra possível evolução ao trabalho seria a implementação de mais funcionalidades ao software. Um exemplo de funcionalidade seria uma evolução ao sistema de sugestão de métrica, que poderia além de sugerir uma métrica, acompanhar outras métricas que não são exibidas no \textit{dashboard}, e quando a métrica alcançar um valor preocupante informar o gestor para que tome uma providencia.