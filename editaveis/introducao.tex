\chapter[Introdução]{Introdução}
%\addcontentsline{toc}{chapter}{Introdução}
	Neste primeiro capítulo, é apresentada uma visão mais ampla do trabalho. Tal visão objetiva introduzir a temática abordada. Este capítulo está dividido em 6 (seis) seções. Na primeira seção, é abordado o contexto (Contextualização) em que se encontra este trabalho. Uma vez contextualizado, apresenta-se a problematica (Problema) . As justificativas que levaram à realização deste trabalho são apresentedas na seção Justificativa deste mesmo capítulo. Após as justificativas, são apresentados os objetivos (Objetivos) que servem como guia para solução proposta. Resultados esperados (Resultados Esperados) Os resultados dessa pesquisa  são apresentados na seção tal, Resultados Esperados. Por fim, tem-se a Organização do Trabalho, a qual procura apresentar os principais capítulos dessa monografia.


\section{Contextualização}
	O conceito de Engenharia de Software foi proposto inicialmente durante uma conferência na década de 60 em Garmisch na Alemanha. Nesta conferência, estavam presentes usuários, fabricantes e pesquisadores que debatiam sobre os constantes problemas no desenvolvimento de software \cite{Paduelli}. Desde então, empresas, orgãos públicos e diversas outras instituições utilizam o computador para automatizar tarefas ou cálculos antes feitos por humanos \cite{fonseca2007historia}. O Decreto n\c 2.271 de 1997 \cite{decreto_2271} coloca a atividade de informática (e seus afins) como sendo um dos tipos de serviço passíveis de tercerização, ou seja, uma empresa tercerizada deve realizar os serviços referentes a este tipo de atividade. Uma vez que o software é produzido por uma empresa tercerizada é necessário que se faça o controle da qualidade deste software.

O controle da qualidade é um dos processos no ciclo de vida de desenvolvimento de software \cite{machado_metricas_2004}. A qualidade de software na produção do software é uma área muito ampla e que abrange desde qualidade da arquitetura de software até qualidade no processo . Este trabalho teve como objetivo central apresentar uma solução para visualização de métricas de qualidade, tendo como os principais pilares três áreas comuns da Engenharia de Software, Gerência de Configuração, Integração Contínua e Análise Estática de Código. Uma solução próxima a essa vem sendo trabalhada dentro de alguns Órgãos Públicos do Governo Federal, entre eles o Tribunal de Contas da União (TCU). Este órgão tem mostrado bons resultados nesta área. O problema encontrado atualmente está na falta de um acompanhamento na qualidade dos softwares entregues pelas empresas tercerizadas 	

\section{Problema de pesquisa}
O principal produto da Engenharia de Software é o software. Contudo o que se tem vivenciado na realidade brasileira de computação é que o software que está sendo entregue é um software precário e de baixa qualidade \cite{schnaider_uma_2004}.

 Por ser um conceito abstrato, qualidade é entendida como algo amplo. Porém, está, normalmente, associada a uma medida relativa. Nesse sentido, qualidade por ser entendida como "conformidade às especificações" \cite{crosby}. Uma vez conceituado dessa forma, a não conformidade às especificações é vista como baixa ou ausência de qualidade.

 Uma das grandes dificuldades nos Orgãos Públicos está no acompanhamento das manutenções prestadas por terceirizadas. Esse problema se agrava ainda mais quando a empresa contratante não consegue acompanhar ou não tem parametros concretos de indicadores de qualidade. 
Este trabalho tem como proposta a criação de um dashboard de monitoramento para softwares legad. Com esse dashboard, pretende-se que seja possível acompanhar mais facilmente, através de recursos visuais, indicadores de qualidade de código nos projetos monitorados.

O Dentre os desafios dessa proposta, tem-se a necessidade de prover uma visualização de dados de forma simplificada e objetiva, visando o acompanhamento dos indicadores de qualidade em um Órgão Público Federal.  Lembrando que, normalmente, os Órgãos Públicos Brasileiros não são os desenvolvedores diretos dos produtos de software. Tais demandas são terceirizadas para empresas especializadas, o que enfatiza a necessidade de se ter um suporte que facilite a avaliação dos entregáveis por parte dos Órgãos Públicos. Tomando este problema como base, tem-se a seguinte questão de pesquisa:
	
	\begin{center}
	\textit{"Definido um conjunto de métricas, como criar um dashboard que avalie a qualidade de software de um órgão público federal ?"}	
	\end{center}

\section{Justificativa}
continuaram. Essas lacunas estão, principalmente, associadas à apresentação
	A motivação deste trabalho deu-se como uma extensão de um trabalho de conclusão de concurso elaborado anteriormente por Luiza e Yago \cite{luiza_yago}. Nesse trabalho, eram tratados aspectos de monitoramento da qualidade, aspectos de contratação de software dentro desses orgãos e como acontecia o acompanhamento desse software. O trabalho também propunha uma solução com integração contínua e gerência de configuração, dentro de um Órgão Público Federal. Após a conclusão do trabalho, algumas lacunas continuaram. Essas lacunas estão, principalmente, associadas à apresentação destes dados para a equipe de gestão, com o intuito de facilitar o acompanhamento das métricas.

Outro trabalho também conferiu base para a proposta. Trata-se do trabalho de conclusão de curso de Adriano [Silva 2014]. Em seu trabalho Adriano apresenta a construção de um painel para acompanhamento de métricas, voltadas para qualidade de software na área de teste de software que seriam implementadas para o laboratório CQTS. Contudo, a presente proposta difere de \cite{silva_painel_2014}, uma vez que são focos: facilitar a comunicação entre as terceirizadas e os gestores de projeto, provendo um suporte que possibilitará a intermediação entre essas partes; e permitir que a qualidade seja aferida pelos gestores de projeto - membros dos Órgãos Públicos - através de um suporte com recursos visuais e orientado por indicadores de qualidade.  

Este trabalho também teve como base outro trabalho de conclusão de concurso de Adriano \cite{silva_painel_2014}, contudo o propósito deste trabalho é apresentar uma visão diferente a já abordada. O foco deste trabalho está em facilitar a comunicação entre a tercerizada e o gestor de projetos apresentando uma ferramenta que fará a ponte entre as duas partes.


	\section{Objetivos}

	\subsection{Objetivos Gerais} % (fold)
	\label{sub:objetivos_gerais}
	
	O objetivo deste trabalho é propor um \textit{dashboard}, que com auxílio de recursos visuais auxilie no processo de contratação de software. O público alvo desta solução são gestores de projeto, e líderes de setores das áreas de tecnologia dos Órgãos Públicos Federais Brasileiros. A solução engloba a Visualização da Informação através de um \textit{dashboard} em que são mostrados métricas e indicadores previamente definidos pelo gestor. Estas métricas são escolhidas pelo gestor de um conjunto de métricas que são apresentadas pela solução. Espera-se que com o \textit{dashboard}, os gestores dos projetos tenham mais visibilidade do processo de construção ou manutenção do software prestado pelas terceirizadas.

	% subsection objetivos_gerais (end)

	\subsection{Objetivos Específicos} % (fold)
	\label{sub:objetivos_específicos}

	Para que seja possível alcancar o objetivo geral, alguns objetivos mais específicos precisam ser levados em consideração. São eles:
		 
	\begin{itemize}
		\item Definir um conjunto de métricas que possam ser utilizadas pelo gestor como indicadores da qualidade de código do software em análise.
		\item Utilizar dados de uma ferramenta de analise estática para coleta de métricas.  
		\item Propor um dashboard de visualização e acompanhamento de qualidade de código, instanciando-o para um projeto específico. Nesse caso, será necessário simular um ambiente, configurado com base em um Órgão Público Federal.
		
	\end{itemize}
	
	% subsection objetivos_específicos (end)
	
	\section{Resultados Esperados}

Ao fim deste trabalho, espera-se apresentar uma solução em software para visualização de métricas coletadas  juntamente com uma ferramenta de analíse estática. Essa visualização ocorrerá através de um \textit{dashboard} que indica a atual situação de um projeto tendo como indicadores métricas estabelecidas por um gestor de tecnologia. 
Além dos resultados abordados, também espera-se alcancar maior conhecimento na área de Qualidade de Software também como nas áreas associadas a esta, por exemplo, Gerência de Configuração, Integração Contínua e Controle de Versão.

	\section{Organização do trabalho} % (fold)
	\label{sec:organização_do_trabalho} 	
	% section organização_do_trabalho (end)
A monografia está organizada em capítulos. No primeiro capítulo, é apresentada uma \textbf{Introdução} ao trabalho, destacando a problemática e os objetivos a serem alcançados. No segundo capítulo, encontra-se o \textbf{Referencial Teórico} serve como base bibliográfica para conceituação de termos que são usados ao longo do trabalho. O terceiro capítulo apresenta uma descrição das \textbf{Ferramentas} que foram utilizadas neste trabalho, destacando, principalmente o SonarQube. No quarto capítulo, é apresentada a \textbf{Proposta} deste trabalho, é neste capítulo que é detalhado um pouco mais a pesquisa em andamento. O quinto capítulo é voltado para a \textbf{Metodologia} a qual foi e será aplicada para que se possa alcançar os objetivos estabelecidos. No sexto e último capítulo, ão apresentados os status das atividades concluídas e/ou em andamento bem como os principais resultados obtidos até o momento.
