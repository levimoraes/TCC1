\chapter[Introdução]{Introdução}
%\addcontentsline{toc}{chapter}{Introdução}
	Neste primeiro capítulo é apresentado uma visão mais ampla do trabalho e que tem como objetivo introduzir a temática abordada. Este capítulo está dividido em 6 (seis) seções. Na primeira seção é abordado o contexto (Contextualização) do cenario em que se encontra este trabalho. Seguido da contextualização apresenta-se a problematica (Problema) que deseja-se resolver com a solução apresentada. As justificativas (Justificativa) que levaram à realização deste trabalho. Após as justificativas são apresentados os objetivos (Objetivos) que servem como guia para solução proposta. Resultados esperados (Resultados Esperados) que como o próprio nome já sugere apresenta o que é esperado ao fim deste trabalho e por último organização do Trabalho (Organização do Trabalho) que descreverá o conteudo apresentado durante todo o trabalho.

\section{Contextualização}
	O conceito de engenharia de software foi proposto inicialmente durante uma conferência na década de 60 em Garmisch na Alemanha. Nesta conferência estavam presentes usuarios, fabricantes e pesquisadores que debatiam sobre os constantes problemas no desenvolvimento de software \cite{Paduelli}.
	

\section{Problema de pesquisa}



\section{Justificativa}

	A utilização da Robótica como uma forma de ensinar programação em escolas e faculdades, conhecida como Robótica Educacional \cite{roboticaEducacionalAulasMatematica}, traz alguns benefícios para o aluno. Conforme colocado pelos autores \cite{teachingWithRoboticKit}, \cite{roboticEducationBasedLego}, \cite{roboticaEducacionalAulasMatematica} e \cite{evaluationRoboticEducationScale}, alguns desses benefícios são: 
	\begin{itemize}
		\item maior interesse pelos conteúdos estudados em aula;
		\item capacidade de trabalhar em grupo;
		\item aplicação prática do conhecimento teórico, e
		\item multidisciplinaridade.
	\end{itemize}
	 


	\section{Objetivos}

	\subsection{Objetivo geral} % (fold)
	\label{sub:objetivos_gerais}
	
		Adaptar técnicas de resolução do problema de SLAM para o contexto de robôs simples, utilizando os kits de robótica \textit{Mindstorms} da Lego, em um primeiro momento.

	% subsection objetivos_gerais (end)

	\subsection{Objetivos específicos} % (fold)
	\label{sub:objetivos_específicos}
		 
	\begin{itemize}
		\item Identificar uma solução para o problema de SLAM em um contexto simplificado;
		\item Propor adaptação para o contexto de robôs simples, na Robótica Educacional;
		\item Implementar adaptação.
	\end{itemize}
	
	% subsection objetivos_específicos (end)



	\section{Resultados Esperados}

	\section{Organização do trabalho} % (fold)
	\label{sec:organização_do_trabalho}
		Este trabalho de conclusão de curso está organizado nos capítulos:
		\begin{itemize}
			\item Introdução: Capítulo referente à contextualização, levantamento da questão de pesquisa, justificativa e definição dos objetivos do trabalho;

			\item Referencial teórico: O objetivo deste capítulo é fornecer ao leitor o conhecimento necessário para compreender a pesquisa realizada. O capítulo é sub-dividido nas seções \textit{robótica e a auto-localização},  \textit{o problema de SLAM} e \textit{robótica educacional};

			\item Suporte tecnológico: Apresenta as ferramentas e tecnologias utilizadas para auxiliar o desenvolvimento desta pesquisa, desde a pesquisa bibliográfica e documentação, até o desenvolvimento da prova de conceito e apresentação;

			\item Metodologia: Este capítulo busca apresentar as técnicas utilizadas para a realização da pesquisa, definindo as atividades a serem desempenhadas para conclusão do trabalho, e

			\item Resultados parciais: Neste capítulo, são apresentados os resultados obtidos durante o desenvolvimento da primeira etapa deste trabalho de conclusão de curso.

			\item Consiferações finais: Capítulo final do TCC\_1 que tem como objetivo apresentar o status atual do trabalho, assim como o que se espera para a próxima etapa do trabalho.
		\end{itemize}
	% section organização_do_trabalho (end)