\chapter[Introdução]{Introdução}
%\addcontentsline{toc}{chapter}{Introdução}
	Neste primeiro capítulo, é apresentada uma visão mais ampla do trabalho. Tal visão objetiva introduzir a temática abordada. Este capítulo está dividido em 6 (seis) seções. Na primeira seção, é abordado o contexto (Contextualização) em que se encontra este trabalho. Uma vez contextualizado, apresenta-se a problemática (Problema) . As justificativas que levaram à realização deste trabalho são apresentadas na seção Justificativa deste mesmo capítulo. Após as justificativas, são apresentados os objetivos (Objetivos) que servem como guia para a solução proposta. Os resultados dessa pesquisa  são apresentados na seção Resultados Esperados. Por fim, tem-se a Organização do Trabalho, a qual procura apresentar os principais capítulos desta monografia.


\section{Contextualização}
	O conceito de Engenharia de Software foi proposto inicialmente durante uma conferência na década de 60 em Garmisch na Alemanha. Nesta conferência, estavam presentes usuários, fabricantes e pesquisadores que debatiam sobre os constantes problemas no desenvolvimento de software \cite{Paduelli}. Desde então, empresas, órgãos públicos e diversas outras instituições, utilizam o computador para automatizar tarefas ou cálculos antes feitos por humanos \cite{fonseca2007historia}. O Decreto 2.271 de 1997 \cite{decreto_2271} coloca a atividade de informática (e seus afins) como sendo um dos tipos de serviços passíveis de terceirização, ou seja, uma empresa terceirizada deve realizar os serviços referentes a este tipo de atividade. Uma vez que o software é produzido por uma empresa terceirizada é necessário que se faça o controle da qualidade deste software.

O controle da qualidade é um dos processos no ciclo de vida de desenvolvimento de software \cite{machado_metricas_2004}. A qualidade na produção de software é uma área muito ampla, e que abrange desde qualidade da arquitetura de software, a qualidade no processo. Este trabalho teve como objetivo central apresentar uma solução para visualização de métricas de qualidade, tendo como os principais pilares três áreas comuns da Engenharia de Software, Gerência de Configuração, Integração Contínua e Análise Estática de Código. Uma solução próxima a essa vem sendo trabalhada dentro de alguns Órgãos Públicos do Governo Federal, entre eles o Tribunal de Contas da União (TCU). Este Órgão tem mostrado bons resultados nesta área. O problema encontrado atualmente está na falta de um acompanhamento na qualidade do software entregues pelas empresas terceirizadas 	

\section{Problema de pesquisa}
Por ser um conceito abstrato, qualidade é entendida como algo amplo. Porém, está, normalmente, associada a uma medida relativa. Nesse sentido, qualidade por ser entendida como "conformidade às especificações" \cite{crosby}. Uma vez conceituado dessa forma, a não conformidade às especificações é vista como a baixa ou ausência de qualidade.

Uma das grandes dificuldades nos Órgãos Públicos está no acompanhamento das manutenções prestadas por empresas terceirizadas. Esse problema se agrava ainda mais quando a empresa contratante não consegue acompanhar ou não tem parâmetros concretos de indicadores de qualidade. Este trabalho tem como proposta a criação de um \textit{dashboard} de monitoramento, para a fase de desenvolvimento do software. Com esse \textit{dashboard}, pretende-se que seja possível acompanhar mais facilmente, através de recursos visuais, indicadores de qualidade de código nos projetos monitorizados.

Dentre os desafios dessa proposta, tem-se a necessidade de prover uma visualização de dados de forma simplificada e objetiva, visando o acompanhamento dos indicadores de qualidade em um Órgão Público Federal.  Lembrando que, normalmente, os Órgãos Públicos Brasileiros não são os desenvolvedores diretos dos produtos de software. Tais demandas são terceirizadas para empresas especializadas, o que enfatiza a necessidade de se ter um suporte que facilite a avaliação dos entregáveis por parte dos Órgãos Públicos. Tomando este problema como base, tem-se a seguinte questão de pesquisa:
	
	\begin{center}
	\textit{"Definido um conjunto de métricas, como criar um \textit{dashboard} que avalie a qualidade de software de um órgão público federal ?"}	
	\end{center}

\section{Justificativa}
	A motivação deste trabalho ocorreu durante um dos projetos em que participei na faculdade.  O projeto se tratava de uma parceria entre a academia e um órgão público federal. Neste projeto existiam diversas áreas de trabalho, sendo uma delas Qualidade de Software, a qual tive a oportunidade de trabalhar, juntamente com outros dois alunos, Luiza Schaidt e Yago Regis.
	
	Pelo fato de as atividades desenvolvidas no projeto afetarem diretamente algumas terceirizadas que tinham contrato juntamente com o órgão, se fazia constante reuniões com as três partes: órgão, academia e terceirizada. Com o decorrer do projeto, fui ganhando maturidade para começar a discernir algumas das práticas adotadas pela empresa contratada. Uma das práticas que mais me chamava atenção era o fato de que o produto de software entregue pela empresa, era um produto de qualidade ruim, que na maioria das vezes dava defeito quando já estava em produção. Comecei a perceber que essa prática ocorria por falta de um controle rigoroso do órgão que na ilusão de ter um software funcionando acabava aceitando o produto entregue ignorando o que estava por trás da funcionalidade entregue.
	
	Desde então, comecei a perceber que o trabalho de implantar um ambiente que incentivasse a produção com qualidade seria ineficaz, enquanto não houvesse um acompanhamento por parte do órgão em verificar a qualidade do software entregue, durante o processo de desenvolvimento pela empresa contratada.


	\section{Objetivos}

	\subsection{Objetivos Gerais} % (fold)
	\label{sub:objetivos_gerais}
	
	O objetivo deste trabalho é propor um \textit{dashboard}, que com auxílio de recursos visuais auxilie no processo de contratação de software. O público alvo desta solução são gestores de projeto, e líderes de setores das áreas de tecnologia dos Órgãos Públicos Federais Brasileiros. A solução engloba a Visualização da Informação através de um \textit{dashboard} em que são mostrados métricas e indicadores baseados no perfil do gestor . Estas métricas são escolhidas com base no alinhamento do gestor. Espera-se que com o \textit{dashboard}, os gestores dos projetos tenham mais visibilidade da qualidade dentro do processo de construção e manutenção do software entregue pelas empresas contratadas.

	% subsection objetivos_gerais (end)

	\subsection{Objetivos Específicos} % (fold)
	\label{sub:objetivos_específicos}

	Para que seja possível alcançar o objetivo geral, alguns objetivos mais específicos precisam ser levados em consideração. São eles:
		
	\begin{itemize}
		\item Utilizar do conceito de Aprendizagem Computacional para sugerir um conjunto de métricas que possam ser utilizadas pelo gestor como indicadores da qualidade de código do software em análise.
		\item Utilizar dados de uma ferramenta de análise estática para coleta de métricas. 
		\item Propor um \textit{dashboard} de visualização e acompanhamento de qualidade de código, instanciando-o para um projeto específico. Nesse caso, será necessário simular um ambiente, configurado com base em um Órgão Público Federal.
		
	\end{itemize}
	
	% subsection objetivos_específicos (end)
	
	\section{Resultados Esperados}

Ao fim deste trabalho, espera-se apresentar uma solução em software, para visualização de métricas coletadas,  juntamente com uma ferramenta de análise estática. Essa visualização ocorrerá através de um \textit{dashboard}, que indica a atual situação de um projeto, tendo como indicadores métricas sugeridas pelo software com base no perfil de administração de um gestor.
Além dos resultados abordados, também espera-se alcançar maior conhecimento na área de Qualidade de Software, também como, nas áreas associadas a esta, por exemplo, Gerência de Configuração, Integração Contínua e Controle de Versão.

	\section{Organização do trabalho} % (fold)
	\label{sec:organização_do_trabalho} 	
	% section organização_do_trabalho (end)
A monografia está organizada em capítulos. No primeiro capítulo, é apresentada uma \textbf{Introdução} ao trabalho, destacando a problemática e os objetivos a serem alcançados. No segundo capítulo, encontra-se o \textbf{Referencial Teórico} serve como base bibliográfica para conceitualização de termos que são usados ao longo do trabalho. O terceiro capítulo apresenta uma descrição das \textbf{Ferramentas} que foram utilizadas neste trabalho, destacando, principalmente o SonarQube. O quarto capítulo é voltado para a \textbf{Metodologia} a qual este trabalho foi aplicado para que se pudesse alcançar os objetivos estabelecidos. No quinto capítulo, é apresentado o \textbf{Dashboard} que é o produto de software deste trabalho, é neste capítulo que se detalha um pouco mais sobre o funcionamento e a avaliação à qual ele foi submetido. No sexto capítulo, são apresentados os resultados das atividades de desenvolvimento e avaliação do produto de software. O sétimo e ultimo capítulo se refere às Conclusões que se chegou com o desenvolvimento deste trabalho.