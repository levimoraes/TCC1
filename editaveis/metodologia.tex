\chapter[Metodologia]{Metodologia}
Este capítulo tem como objetivo apresentar a metodologia utilizada para desenvolvimento do TCC 1, e mostrar o plano base que servirá como guia para o desenvolvimento do TCC 2. A seção \ref{met_pesquisa} apresenta a classificação da metodologia utilizada para o desenvolvimento do trabalho juntamente com o plano metodológico. A seção \ref{met_desenvolvimento} descorre sobre o processo de desenvolvimento utilizado na criação da solução. E por último a seção  \ref{cronograma} apresenta o cronograma utilizado no desenvolvimento do TCC 1 é um cronograma para o TCC 2.
\section{Metodologia de Pesquisa}
\label{met_pesquisa}
Segundo \cite{moresi_metodologia_2003} a pesquisa pode ser entendida como um conjunto de ações que tem como objetivo encontrar um problema, construida através de procedimentos empíricos e sistemáticos. Para Moresi existem quatro classificações básicas para a pesquisa, quanto a sua natureza, sua abordagem, seu objetivo e o meio pelo qual é feita a investigação. A figura \ref{img:met_pesquisa} apresenta em quais características este trabalho se apresenta.
\graphicspath{{figuras/}}
\begin{figure}[h]
\centering
\includegraphics[scale=0.50]{metodologia_pesquisa}
\caption{Seleção das Características Metodológicas. Fonte: \cite{moresi_metodologia_2003}}
\label{img:met_pesquisa}
\end{figure}

Este trabalho tem um caráter mais voltado para uma pesquisa aplicada por envolver características espécificas na contratação de software do governo brasileiro. Outra característica que determina este trabalho como uma pesquisa aplicada é a natureza do trabalho estar voltado para o uso nas áreas de TI dentro dos orgãos públicos.
\\Segundo \cite{tatiana_denise} a pesquisa qualitativa é mais voltada para aspectos da realidade que não podem ser quantificados, mantendo o foco na compreensão. Neste aspecto o trabalho apresenta características qualitativas uma vez que a construção da solução é feita de maneira dinâmica se adaptando as necessidades do usuário. Segundo as autoras outra característica inerente a este tipo de pesquisa é a observação do mundo social ao mundo natural, está característica se apresenta de maneira muito forte quando propor a solução foi adotado um conjunto de métricas que são utilizadas no mercado ao invés de outros conjuntos apresentados por outros autores. 
\\Para Gil \cite{gil_como_2002} a pesquisa descritiva é focada em analisar características de uma população, fenômeno ou a relação entre as váriaveis as que compõem. Ests tipo de pesquisa não visa explicar os fenômenos que está descrevendo (apesar que a explicação para o fenômeno possa servir como base) \cite{moresi_metodologia_2003}. Este trabalho apresenta o caráter descritivo ao se tratar da natureza das contratações de software para orgão brasileiros, ou quando se fala de um processo de manutenção de software.
\\Outra característica deste trabalho é o fato de se tratar de uma pesquisa 	de laboratório. Moresi destaca que a pesquisa de laboratório atua em um ambiente controlavel em que o pesquisador não tem a possibilidade de atuar em campo. Neste trabalho a pesquisa em campo se torna algo complicado pois é difícil de se conseguir acesso a orgãos públicos para instalação de uma ferramenta que ainda está em desenvolvimento. Por este motivo a pesquisa em laboratório é a mais adequada, em que são recriados as mesmas condições de uma situação em campo, porém com o controle de um ambiente simulado.

\subsection{Plano Metodológico}
\label{plano_metodologico}
O plano metodológico consiste em duas fases: Iniciação e Execução. Durante o TCC1 será implementado somente a primeira fase e a fase de Execução será implementadaß no TCC2. A primeira fase (imagem \ref{img:iniciacao}) possui quatro atividades principais, são elas: Elaborar um roteiro de pesquisa, Pesquisar Referencias, Refinar Pesquisa e Catalogar material encontrado. 
\\A atividade de Elaboar um Roteiro de Pesquisa consiste em encontrar a melhor string de busca. As strings foram montadas de acordo com o tema da pesquisa então não houve uma string geral que perpassa-se por todo o conhecimentodo trabalho. Essa escolha se deu pelo fato de que este trabalho passou por adaptações até que se chegasse ao trabalho que é hoje, então as primeiras strings eram feitas com conceitos chaves e já definidos sobre o que seria o trabalho, e com o decorrer o trabalho foi ganhando um escopo mais conciso.
\\A atividade de Elaborar Referencias e Refinar Pesquisa envolviam o processo de aplicar a string nas motores de busca selecionados, que no caso foram Google Scholar e Periódicos Capes. Uma vez aplicada a string o primeiro ponto a ser observado nos resultados era o título do material, caso o titúlo tivesse alguma relação com o tema pesquisado o artigo era separado (esse processo era válido somente para as duas primeiras páginas de resultados). Com os artigos separados lia-se os tópicos do artigo e havendo um conteúdo referente à pesquisa lia-se o artigo completo. Uma vez que era feita essa pesquisa gerava-se uma nova string de busca com termos aproximados ou mais refinados e o processo se repetia.
\\Catalogar material é uma atividade focada em guardar os materiais encontrados colocando uma tag referente ao tema a que o artigo se refere e uma breve descrição sobre o que era mais importante. Esse armazenamento é feito através de duas ferramentas de gerenciamento bibliográfico, o Zotero e o BibDesk. O Zotero foi utilizado para fazer a catalogação online dos materiais e gerar a bibliografia encontrada de cada material como mostra na figura \ref{img:zotero}.
\begin{itemize}
\item Área 1 - Categorização por pastas dos artigos encontrados.
\item Área 2 - Artigos referentes à categoria seleciona. Dentro de cada artigo é possível encontrar a nota e um link para leitura do artigo selecionado.
\item Área 3 - Informações do artigo selecionado.
\item Área 4 - Tags referentes ao artigo.
\end{itemize}
Uma vez que o material era catalogado no Zotero ele era exportado para o Bibdeks por ter uma melhor integração com o Latex.
\graphicspath{{figuras/}}
\begin{figure}[h]
\centering
\includegraphics[scale=0.50]{zotero_edit2}
\caption{\textit{Screenshot} do Zotero contendo as categorias dos materiais pesquisados, suas \textit{tags} e anotações}
\label{img:zotero}
\end{figure}

\graphicspath{{figuras/}}
\begin{figure}[h]
\centering
\includegraphics[scale=0.50]{iniciacao}
\caption{Modelagem da Fase de Iniciação}
\label{img:iniciacao}
\end{figure}


A Segunda fase consiste em estudar o cenario em que será colocado a dashboard para que se possa entender quais as necessidades e os requisitos para implantação. Esta fase está dividida em três momentos: planejar,  executar e checar. Durante o momento de planejar tem-se como principais atividades analisar o problema onde há um maior entendimento do contexto no qual será elaborado a solução. A segunda atividade é elaborar solução em que é esperado que ao fim dessa atividade exista um esboço do que será a solução final. No segundo momento implementa-se a solução e no terceiro momento acontece a validação dessa solução.
\graphicspath{{figuras/}}
\begin{figure}
\centering
\includegraphics[scale=0.40]{TCCMetodologia}
\caption{Plano de Pesquisa}
\label{Rotulo}
\end{figure}

\section{Metodologia de Desenvolvimento}
\label{met_desenvolvimento}
O dashboard será criado utilizando de uma adaptação da metodologia de desenvolvimento Scrum. O Scrum é uma metodologia de desenvolvimento de software baseada em princípios de desenvolvimento ágil. As metodologias ágeis tem sido muito utilizadas em projetos 	com times pequenos, curto prazo de entrega do software e os requisitos são constantemente alterados \cite{lopez-martinez_problems_2016}. Como o Scrum é um modelo de desenvolvimento iterativo e incremental ele quebra o projeto de desenvolvimento em pequenas entregas chamadas de \textit{sprints} que seriam 	pequenos ciclos de desenvolvimento que duram entre duas a quatro semanas. As \textit{sprints} são consecutivas e nesse período algumas funcionalidades do sistema são implementadas e testadas \cite{pagotto_scrum_2016}. Na figura \ref{img:scrum} pode-se observar que o as funcionalidades desenvolvidas na \textit{sprint} advêm de um escopo definido no início do projeto chamado de \textit{product backlog} que é o conjunto de todas as funcionalidades do sistema, o conjunto de funcionalidades separadas para uma determinada \textit{sprint} é chamada de \textit{sprint backlog}\cite{sabbagh_scrum:_2014}.
\graphicspath{{figuras/}}
\begin{figure}[h]
\centering
\includegraphics[scale=0.40]{scrum}
\caption{Ciclo de Desenvolvimento do Scrum}
\label{img:scrum}
\end{figure}

Os principais conceitos que foram utilizados da metodologia foram:
\begin{itemize}
\item \textit{\textbf{Product Backlog}}: Lista de atividades que representam as funcionalidades que serão construidas no projeto. O \textit{Product Owner} é quem escreve o \textit{product backlog}, essas atividades são mutáveis ao decorrer do projeto, pois a equipe de desenvolvimento acaba conhecendo mais do produto \cite{sabbagh_scrum:_2014}.
\item \textit{\textbf{Sprint}}: Ciclo de desenvolvimento com prazo definido em que são desenvolvidas as atividades do projeto. Neste trabalho definiu-se como sendo 15 dias o período referente a uma \textit{Sprint}
\item \textit{\textbf{Sprint Backlog}}: Atividades referentes à uma determinada \textit{Sprint} na qual a equipe de desenvolvimento se compromete a entregar. Estas atividades são retiradas do \textit{product backlog} \cite{mahnic_case_2011}.
\item \textbf{História do Usuário}: Descrição do ponto de vista do usuário sobre uma funcionalidade do produto. Este artefato é composto do que é conhecido como 3C's, Cartão, Conversa e Confirmação. O cartão é referente ao fato de se documentar a conversa em um cartão, este sendo acessível a todo o time de desenvolvimento. A conversa é uma breve descrição da funcionalidade sob o olhar do usuário, um exemplo pode ser observado na figura \ref{img:us}. E a confirmação ou critérios de aceitação é um \textit{checklist} abordando o que deve ser verificado para que a história seja dada como concluida.
\graphicspath{{figuras/}}
\begin{figure}[h]
\centering
\includegraphics[scale=0.80]{US}
\caption{Exemplo de História de Usuário. Fonte: \cite{sabbagh_scrum:_2014}}
\label{img:us}
\end{figure}
\item \textit{\textbf{Story Point}}:Unidade relativa que caracteriza o esforço da equipe de desenvolvimento para finalizar uma atividade. A escala utilizada é definida pela equipe de desenvolvimento, neste trabalho será utilizado a escala de Fibonacci (1,2,3,5,8,13, ...). 
\item \textit{\textbf{Kanban}}: Forma de visualização de atividades muito utilizadas com cartões que são movimentados em um quadro determinando o status da atividade como mostra a figura \ref{img:kanban}.
\graphicspath{{figuras/}}
\begin{figure}[h]
\centering
\includegraphics[scale=0.40]{kanban}
\caption{Exemplo de \textit{Kanban}}
\label{img:kanban}
\end{figure}

\end{itemize}
\section{Cronograma}
\label{cronograma}
Para que se possa ter uma visão mais abrangente da organização do trabalho foi criado um cronograma em que constam as atividades definidas no \ref{plano_metodologico}. Este cronograma serve como orientação ao desenvolvimento e planejamento do projeto contudo com o decorrer das atividades este artefato pode sofrer alterações. O cronograma foi dividido em duas tabelas referentes às atividades do TCC1 e TCC2.

\subsection{Cronograma TCC1}
\begin{table}[http]
	\centering
	\caption{Cronograma TCC 1}
	\label{tab:cronograma}
	\begin{tabular}{ccccc}
		\hline
		\multicolumn{1}{|c|}{\textbf{Cronograma}}             & \multicolumn{1}{c|}{\textbf{Agosto}} & \multicolumn{1}{c|}{\textbf{Setembro}} & \multicolumn{1}{c|}{\textbf{Outubro}} & \multicolumn{1}{c|}{\textbf{Novembro}} \\ \hline
		\multicolumn{1}{|c|}{Realizar Pesquisa Bibliográfica} & \multicolumn{1}{c|}{X}              & \multicolumn{1}{c|}{X}              & \multicolumn{1}{c|}{X}             & \multicolumn{1}{c|}{X}              \\ \hline
		\multicolumn{1}{|c|}{Estudar o orgão}             & \multicolumn{1}{c|}{}               & \multicolumn{1}{c|}{X}              & \multicolumn{1}{c|}{X}             & \multicolumn{1}{c|}{}               \\ \hline
		\multicolumn{1}{|c|}{Propor Versão Inicial do dashboard}                & \multicolumn{1}{c|}{}               & \multicolumn{1}{c|}{}               & \multicolumn{1}{c|}{X}             & \multicolumn{1}{c|}{X}              \\ \hline
		\multicolumn{1}{l}{}                                  & \multicolumn{1}{l}{}                & \multicolumn{1}{l}{}                & \multicolumn{1}{l}{}               & \multicolumn{1}{l}{}                \\
		\multicolumn{1}{l}{}                                  & \multicolumn{1}{l}{}                & \multicolumn{1}{l}{}                & \multicolumn{1}{l}{}               & \multicolumn{1}{l}{}                \\
		\multicolumn{1}{l}{}                                  & \multicolumn{1}{l}{}                & \multicolumn{1}{l}{}                & \multicolumn{1}{l}{}               & \multicolumn{1}{l}{}               
	\end{tabular}
\end{table}

\subsection{Cronograma TCC2}

\section{Resumo do Capítulo}
A pesquisa pode ser classificada de diversas formas, neste a projeto a natureza da pesquisa pode ser classificada como Aplicada devido ao uso nas áreas de TI. Quanto a sua abordagem ela é do tipo Qualitativa pois a forma de se analisar e coletar os dados é feito de maneira empírica e através da observação. Sobre o seu objetivo essa pesquisa tem um caráter Descritivo pois ela observa fatores de um grupo e os descreve neste caso a maneira como funciona a aquisição de software por parte da APF. E por último quanto ao meio de investigação que Laboratorial pois todo o desenvolvimento da pesquisa é feita em um ambiente controlado e não em campo.
Quanto a metodologia de desenvolvimento optou-se por uma adaptação da metodologia Scrum, isto se deu pelo fato de que o framework da metodologia é adaptável para times de desenvolvimento pequenos e com entregas de software em curtos períodos de tempo.
