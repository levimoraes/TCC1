%\part{Aspectos Gerais}

\chapter[Referencial Teórico]{Referencial teórico}
	
	Este capítulo tem como objetivo servir como referencial teórico para todo o documento. As idéias discutidas neste capítulo são
	
\section{Qualidade}
	O principal produto da engenharia de software é o software, contudo o que tem se vivenciado na realidade brasileira de computação é que o software que está sendo entregue é um software precário e de baixa qualidade. Por ser uma palavra abstrata, o conceito de qualidade é bem amplo, porém o termo qualidade normalmente está associado a uma medida relativa, essa qualidade pode ser entendida como "conformidade às especificações". Conceituando dessa forma, a não conformidade às especificação é igual a ausência de qualidade \cite{Paduelli}.
\\ A ISO 9126-1 proposta em 2001, também conhecida como Engenharia de Software - Qualidade do Produto, descreve o modelo de qualidade voltado para o produto de software como sendo composto por duas categorias como pode ser visto na Figura \ref{img:relacao_iso}. A primeira categoria está relacionada a qualidade interna e a qualidade externa do software. A segunda categoria se relaciona com a qualidade de uso do software \cite{_nbr_2016}
\graphicspath{{figuras/}}
\begin{figure}
\centering
\includegraphics[scale=0.40]{ISO}
\caption{Relação entre as NBR ISO/IEC 9126 e NBR ISO/IEC 14598 .Fonte:\cite{_nbr_2016}}
\label{img:relacao_iso}
\end{figure}
\\ Sob o aspecto de modelo de qualidade,a ISO 9126 classifica a qualidade interna do produto como sendo o somatório das características do ponto de vista interno do software. Os principais produtos desta categoria são os de cunho intermediário, entre eles: relatórios de analise estática do código fonte, revisão dos documentos produzidos, entre outros. A qualidade externa por sua vez já apresenta o seu foco mais voltado para as relações externas do software, normalmente esta relacionado com a execução do código coletando suas métricas enquanto o software está em funcionamento. A Figura \ref{img:modelo_qualidade} apresenta a divisão proposta pela \cite{_nbr_2016} onde são categorizados seis aspectos de qualidade de software e suas subcaracterísticas, essas podem ser medidas por meio de métricas internas e externas. 
 \graphicspath{{figuras/}}
\begin{figure}
\centering
\includegraphics[scale=0.50]{Modelo_de_Qualidade}
\caption{Modelo de Qualidade para Qualidade Interna e Externa .Fonte:\cite{_nbr_2016}}
\label{img:modelo_qualidade}
\end{figure}
\\Segundo a ISO 9126 essas caracteristicas podem ser definidas como:
\begin{itemize}
\item \textbf{Funcionalidade}: Capacidade do produto de software de prover funções que atendam às necessidades explícitas e implícitas, quando o software estiver sendo utilizado sob condições especificadas.
\item \textbf{Confiabilidade}:Capacidade do produto de software de manter um nível de desempenho especificado, quando usado em condições especificadas.
\item \textbf{Usabilidade}: Capacidade do produto de software de ser compreendido, aprendido, operado e atraente ao usuário, quando usado sob condições especificadas.
\item \textbf{Eficiência}: Capacidade do produto de software de apresentar desempenho apropriado, relativo à quantidade de recursos usados, sob condições especificadas.
\item \textbf{Manutenibilidade}: Capacidade do produto de software de ser modificado. As modificações podem incluir correções, melhorias ou adaptações do software devido a mudanças no ambiente e nos seus requisitos ou especificações funcionais.
\item \textbf{Portabilidade}: Capacidade do produto de software de ser transferido de um ambiente para outro.
\end{itemize}
Em 2011 surgiu um conjunto de normas conhecidos como SQuaRE que traziam um framework aprimorado à atual norma vigente. Este framework tinha como objetivo avaliar o produto de qualidade de software.

\subsection{Normas SQuaRE}
Este conjunto de normas surgiu para substituir a ISO/IEC 9126, Engenharia de Software - Qualidade do Produto. ISO/IEC 25010 mantém as características de qualidade com alguns incrementos.

\begin{itemize}
\item O escopo dos modelos de qualidade foram extendidos para incluir sistemas computacionais e a qualidade em uso pelo ponto de vista do sistema
\item Segurança foi adicionada como característica, e não uma subcaracterística de funcionalidade.
\item Compatibilidade foi adicionada como característica.
\item A qualidade interna e externa foram combinadas como modelo de qualidade de produto.
\end{itemize}

A norma apresenta três guias de qualidade. O primeiro modelo é referente à Qualidade do Produto, o segundo da Qualidade em Uso e o último Qualidade de Dados.O modelo de Qualidade do Produto subdivide um sistema de software em oito categorias como mostra a imagem \ref{img:modelo_square}.
\graphicspath{{figuras/}}
\begin{figure}
\centering
\includegraphics[scale=0.40]{SQuaRE}
\caption{Produto de Qualidade de Software.Fonte:\cite{Square}}
\label{img:modelo_square}
\end{figure}

Este trabalho tem seu desenvolvimento focado no modelo de Qualidade de Uso que apresenta características externas ao software \cite{Square}. O foco deste trabalho está em medir indicadores quanto à manutenabilidade do software. Essa característica está diretamente ligada ao processo de Manutenção do Software.  

\subsection{Manutenção de Software}
Segundo Sommervile \cite{sommervile} manutenção de software é o processo de alterar o sistema depois que ele foi publicado. As alterações feitas no software podem ser simples correções de erro, a até mudanças significativamente grandes que corrigem falhas arquiteturais, ou mesmo melhorias para acomodar novos requisitos.
\\Outra visão sobre manutenção de software é dada por Pressman \cite{pressman} em que o autor conceitua o termo como sendo a correção de defeitos, adaptação do software para atender uma mudança do ambiente e aperfeiçoar as funcionalidades para que atendam às necessidades dos usuários. Outra característica do processo de manutenção é a sua composição por um conjunto de sub processos, atividades e tarefas que podem ser utilizados durante a fase de manutenção para alterar um produto de software, contanto que seja mantido o seu funcionamento \cite{calazans_avaliacao_2005}.
\\Para Sommervile existem quatro categorias de manutenção:
\begin{itemize}
\item \textbf{Manutenção Corretiva}: seu objetivo está em identificar e remover falhas de software
\item \textbf{Manutenção Adaptativa}: provê modificações no software para alojar mudanças no ambiente externo. Nesta manutenção também está incluso o processo de migração para diferentes plataformas tanto de software quanto de hardware.
\item \textbf{Manutenção Perfectiva}: está manutenção é feito com o intuito de aperfeiçoar o software, além dos requisitos funcionais originais. Esta expansão dos requisitos traz consigo uma melhoria às funcionalidades até então implementadas ou um ganho de performance do sistema.
\item \textbf{Manutenção Preventiva}: implementada para permitir que seja mais simples a correção, adaptação ou melhoria do software.
\end{itemize}

O modelo da figura \ref{img:modelo_manutencao} apresenta as atividades propostas por Pfleeger para um processo de manutenção. Na figura percebe-se que o processo de acompanhamento da manutenção ocorre durante todo o processo. As atividades apresentadas no diagrama são:

\begin{itemize}
\item \textbf{Analise do Impacto da Mudanção de Software}: estima o impacto de uma determinada mudança. Nesta atividade determina-se o grau de mudança e o quanto está mudança impactará no resto do software. 
\item \textbf{Entendimento do Software a ser Alterado}: nesta atividade são analisados os códigos-fonte do software para entender a mudança e a integração do que deve ser alterado. Esta atividade depende muito do grau de manutenabilidade do software, uma vez que quanto mais manutenível mais fácil e rapido se dá o processo de analise do software.
\item \textbf{Implementação da Mudança}: incremento ou modificação do software. Está atividade é diretamente relacionada com o grau de adaptação do software, o quanto o software pode ser expandido ou comprimido. Essa característica de adaptabilidade é uma subcaracterística da Manutenabilidade de software apresentada pela norma Square. 
\item \textbf{Mudanças pelo Efeito Cascata}: Analise da propagação das mudanças ao longo do software. Essa atividade está intimamente relacionada com o indicador de coesão e acomplamento do software, este afere o quão amarrado estão as classes e os métodos do software.
\item \textbf{(Re)Teste do Software}: é a ultima atividade antes da entrega do software alterado. O software é testado novamente sob a perspectiva do novo requisito.
\end{itemize}

\graphicspath{{figuras/}}
\begin{figure}
\centering
\includegraphics[scale=0.50]{Manutencao}
\caption{Diagrama das Atividades de Manutenção de Software.Fonte:\cite{pfleeger_framework_1990}}
\label{img:modelo_manutencao}
\end{figure}
Um estudo realizado por Kustyers e Heemstra \cite{kusters} mostram a dificuldade de atuais na manutenção de software em seis grandes organizações da Alemanha. Um dos resultados obtidos foi que existe um uma falta muito grande na percepção quanto ao tamanho e o custo das manutenções de software. Os autores relatam que os gastos com manutenção são altos e que das seis empresas apenas uma mantinha registrado os seus processos de manutenção e os usava para fazer um novo planejamento. 
\\Normalmente tem se como verdade de que a manutenção de software está unicamente ligada ao conserto de \textit{bugs}, entretanto estudos e \textit{surveys} ao longo dos anos comprovam que mais de 80\% do esforço gasto na manutenção é utilizado em ações não corretivas segundo Pigosky \cite{pigosky}. O autor também afirma que entre 40\% a 60\% do esforço de manutenção está em entender o software que será modificado.
 
\section{Processo de Contratação de Software na APF}
O Decreto n\c 2.271 de 1997 \cite{decreto_2271} dispõe sobre a contratação de serviços pela APF. Segundo o Decreto todos os produtos ou serviços que não apresentam relação direta com o propósito da instituição do governo federal devem ser tercerizados. O Decreto coloca como exemplo algumas atividades como por exemplo, conservação, limpeza, segurança, vigilância, transportes, informática, copeiragem, recepção, reprografia, telecomunicações são atividades livres para tercerização.
\\A licitação é um conjunto de processos administrativos de caráter formal para as compras de bens ou serviços nos governos federais, estaduais ou municipais. Segundo a Lei  n 8.666 de 1993, o governo brasileiro para garantir a isonomia (princípio geral do direito segundo o qual todos são iguais perante a lei) diz que para contratações de bens ou serviços deve-se dar prioridade para licitação \cite{Lei_1993}. A licitação pode ocorrer de quatro categorias: concorrência, tomada de preços, convite e pregão \cite{brazil_licitacoes_2010}.
\\Uma vez que uma empresa ganha a licitação para um serviço a empresa ganha o direito de contratação para prestar o serviço ao orgão contratante. Para ajudar no processo de contratação de empresas terceirizadas voltadas para a área de TI o TCU disponibiliza um Guia de boas práticas de contratações em soluções de TI \cite{guia_boas_praticas}. Segundo o guia a prática de contratação do serviço de desenvolvimento de um sistema de informação pode englobar elementos do tipo:
\begin{itemize}
\item Os softwares do sistema, devidamente documentados e com evidências de que foram testados;
\item As bases de dados do sistema, devidamente documentadas;
\item O sistema implantado no ambiente de produção do órgão;
\item A tecnologia do sistema transferida para a equipe do órgão, que deve ocorrer ao longo de todo o contrato;
\item As rotinas de produção do sistema, devidamente documentadas e implantadas no ambiente de produção do órgão;
\item As minutas dos normativos que legitimem os atos praticados por intermédio do sistema;
\item O sistema de indicadores de desempenho do sistema implantado, que pode incluir as atividades de coleta de dados para gerar os indicadores, fórmula de cálculo de cada indicador e forma de publicação dos indicadores. Citam-se, como exemplos, os indicadores de disponibilidade, de desempenho das transações e de satisfação dos usuários com o sistema de informação;
\item Os scripts necessários para prover os atendimentos relativos ao sistema por parte da equipe de atendimento aos usuários, devidamente implantados e documentados;
\item A capacitação dos diversos atores envolvidos com o sistema (e.g. equipe de suporte técnico do órgão, equipe de atendimento aos usuários, equipe da unidade gestora do sistema e usuários finais), que pode envolver treinamentos presenciais e a distância;
\item O serviço contínuo de suporte técnico ao sistema (e.g. atendimento aos chamados feitos pelo órgão junto à contratada sobre dúvidas e problemas relativos ao sistema);
\item O serviço contínuo de manutenção do sistema (e.g. implantação de manutenções corretivas e evolutivas).
\end{itemize} 