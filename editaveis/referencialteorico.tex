%\part{Aspectos Gerais}

\chapter[Referencial Teórico]{Referencial teórico}
	
	Este capítulo tem como objetivo servir como referencial teórico para todo o documento. As idéias discutidas neste capítulo são
	
\section{Qualidade}
	O principal produto da engenharia de software é o software, contudo o que tem se vivenciado na realidade brasileira de computação é que o software que está sendo entregue é um software precário e de baixa qualidade. Por ser uma palavra abstrata, o conceito de qualidade é bem amplo, porém o termo qualidade normalmente está associado a uma medida relativa, essa qualidade pode ser entendida como "conformidade às especificações". Conceituando dessa forma, a não conformidade às especificação é igual a ausência de qualidade \cite{Paduelli}.
\\ A ISO 9126 descreve o modelo de qualidade como sendo composto por duas categorias. A primeira categoria está relacionada a qualidade interna e a qualidade externa do software. A segunda categoria se relaciona com a qualidade de uso do software \cite{_nbr_2016}
 