%\part{Aspectos Gerais}

\chapter[Referencial Teórico]{Referencial teórico}
	
	Este capítulo tem como objetivo servir como referencial teórico para todo o documento. As idéias discutidas neste capítulo são
	
\section{Qualidade}
	O principal produto da engenharia de software é o software, contudo o que tem se vivenciado na realidade brasileira de computação é que o software que está sendo entregue é um software precário e de baixa qualidade. Por ser uma palavra abstrata, o conceito de qualidade é bem amplo, porém o termo qualidade normalmente está associado a uma medida relativa, essa qualidade pode ser entendida como “conformidade às especificações”. Conceituando dessa forma, a não conformidade às especificação é igual a ausência de qualidade \citep{Paduelli}.
 
 \\ A qualidade de um produto ou serviço, diferente do que muitos imaginam, não é definida pelo programador ou pelo engenheiro e sim pelo consumidor. Essa qualidade é medida com a experiência do usuário. A partir deste outro ponto de vista a qualidade é o conjunto de características que atendem as necessidades do cliente \citep{armenise_continuous_2015}.