\chapter[Resultados parciais]{Resultados Parciais}
\label{sec:resultados_parciais}
Neste capítulo, estão descritos os resultados que foram obtidos até o momento. No caso, tem-se duas seções: a seção \ref{status} que apresenta o estado do projeto atual, e a seção \ref{futuro} que aborda quais serão os próximos passos no desenvolvimento do trabalho.

\subsection{\textit{Status} Atual do Projeto}
\label{status}

Como havia sido definido no capítulo de Introdução, este trabalho tem como objetivo principal desenvolver uma solução em software para monitoramento da qualidade de código. Tomando esse objetivo como linha guia para a elaborações e o desenvolvimento desse projeto, as primeiras atitudes tomadas foram aprofundar mais o conhecimento no assunto e garantir que será possível importar as métricas extraídas pelo SonarQube.

A aquisição de conhecimento foi feita através de um levantamento bibliográfico em várias bases de pesquisa e que influenciaram na escrita do capítulo de Referências Bibliográficas e na tomada de algumas decisões acerca do trabalho, decisões como a escolha da suíte de métricas e as ferramentas para plotagem dos gráficos.

Quanto à importação das métricas, a forma a ser utilizada será de leitura da tela. Foi criado um \textit{script} em \textit{python} que faz a leitura das informações da tela e separa as mesmas por métricas e por projeto. Esse \textit{script} será responsável por coletar e identificar as métricas a serem escolhidas pelo gestor de projetos. 

A Tabela \ref{table_status} mostra a completude do trabalho desenvolvido até o momento, separando por atividades e os meses em que se estima que essas atividades ocorram. Pela tabela é possível perceber que todas as atividades que haviam sido planejadas para o TCC1 já foram concluídas, e a atividade de Analisar Ambiente já foi iniciada devido a instalação do SonarQube. A atividade de Documentação encontra-se em 50\%, pois se refere à documentação entregue enquanto TCC1.

\begin{table}[h!]
\centering
\caption{\textit{Status} Completude do Trabalho}
\label{table_status}
\begin{tabular}{lc}
\rowcolor[HTML]{9A0000} 
{\color[HTML]{FFFFFF} \textbf{Atividade}} & \multicolumn{1}{l}{\cellcolor[HTML]{9A0000}{\color[HTML]{FFFFFF} \textbf{Status}}} \\ \hline
Definir Tema                              & 100\%                                                                              \\ \hline
Validar Escopo                            & 100\%                                                                              \\ \hline
Elaborar Roteiro de Pesquisa              & 100\%                                                                              \\ \hline
Pesquisar Referência                      & 100\%                                                                              \\ \hline
Refinar Pesquisa                          & 100\%                                                                              \\ \hline
Catalogar Material                        & 100\%                                                                              \\ \hline
Documentar                                & 50\%                                                                               \\ \hline
Analisar Ambiente                         & 20\%                                                                               \\ \hline
Configurar Ambiente                       & 0\%                                                                                \\ \hline
Implementar Solução de Software           & 0\%                                                                                \\ \hline
Implantar Solução em Ambiente Simulado    & 0\%                                                                                \\ \hline
Acompanhar Utilização do Software         & 0\%                                                                                \\ \hline
Aplicar Questionário                      & 0\%                                                                                \\ \hline
\end{tabular}
\end{table}


\subsection{Próximas Atividades}
\label{futuro}
Uma vez que já se consegue extrair as informações do SonarQube, a próxima etapa é a de construir o \textit{dashboard}. Para isso serão levantadas todas as funcionalidades que o \textit{dashboard} deve apresentar. Este conjunto de atividades irá compor o \textit{product backlog}. Com o \textit{product backlog} definido, é necessário que se faça a divisão das \textit{sprints} de acordo com o número de funcionalidades. Todas essas atividades fazem parte das atividades necessárias para construção do software.
