\chapter[Resultados]{Resultados}
\label{sec:resultados}
Neste capítulo, estão descritos os resultados que foram obtidos até o momento de publicação deste trabalho. O capítulo apresenta duas seções: a seção \ref{resultados_obtidos} que apresenta os resultados que se obteve ao fim deste trabalho, e a seção \ref{futuro} que aborda possíveis trabalhos que podem dar prosseguimento a este.

\subsection{Resultados Obtidos}
\label{resultados_obtidos}

Após feito o levantamento bibliográfico começou-se a primeira \textit{sprint} do desenvolvimento da solução, que consistia em implementar a funcionalidades: cadastrar um gestor (sendo que este tivesse como realizar um \textit{login} na aplicação) e cadastrar projetos. Com a implementação feita, realizava-se um teste de usabilidade com um grupo de pessoas. Esse grupo de pessoas consistia em quatro alunos de Engenharia de Software da Universidade de Brasília, sendo dois alunos no último semestre do curso, 1 aluno no quinto período e outro do sétimo período. O teste de usabilidade consistia em uma lista de sete atividades que deveriam ser executadas pelo examinado. Nesse primeiro teste foram pedidos que os examinados realizassem atividades mais simples para se contextualizar com o software e com a finalidade do software. Através do gráfico da Figura \ref{img:grafico_iteracao1}, fica possível perceber que os entrevistados tiveram problema para realizar a ativdade 5. Maiores detalhes quanto ao tempo de cada entrevistado se encontram na Tabela \ref{tabela_iteracao1}.

\graphicspath{{figuras/}}
\begin{figure}[!h]
\centering
\includegraphics[scale=0.75]{iteracao1_grafico}
\caption{Gráfico Comparativo das Atividades Realizadas na Iteração 1}
\label{img:grafico_iteracao1}
\end{figure}

\begin{table}[h!]
\centering
\caption{Tempo de Execução de Cada Atividade em Segundos 1a Iteração}
\label{tabela_iteracao1}
\begin{tabular}{|llllll|}
\hline
\multicolumn{6}{|c|}{\cellcolor[HTML]{C0C0C0}\textbf{1a Iteracão}}                     \\ \hline
                   & \textbf{T1} & \textbf{T2} & \textbf{T3} & \textbf{T4} & \textbf{T5} \\ \hline
\textbf{Usuário 1} & 0.3         & 0.31        & 1.41        & 0.5         & 1           \\ \hline
\textbf{Usuário 2} & 0.2         & 0.1         & 1.2         & 0.5         & 1           \\ \hline
\textbf{Usuário 3} & 0.12        & 0.7         & 1.3         & 0.3         & 1           \\ \hline
\textbf{Usuário 4} & 0.12        & 0.3         & 1.2         & 0.4         & 1           \\ \hline
\textbf{Usuário 5} & 0.2         & 0.22        & 1.26        & 0.3         & 1           \\ \hline
\textbf{Usuário 6} & 0.2         & 0.15        & 1.2         & 0.3         & 1           \\ \hline
\textbf{Usuário 7} & 0.2         & 0.18        & 1.34        & 0.3         & 1           \\ \hline
\end{tabular}
\end{table}


Após as atividades propostas pelo entrevistador, perguntava-se aos entrevistados quanto à experiencia que eles haviam tido ao utilizar o software. Um dos pontos ressaltados pelos entrevistados foi quanto ao uso de não haver uma descrição quanto a funcionalidade de voltar ao menu utilizando o canto esquerdo superior da tela como pode ser observado na Figura \ref{img:alteracao_menu}. 

\graphicspath{{figuras/}}
\begin{figure}[h!]
\centering
\includegraphics[scale=0.80]{comparacao_versoes_menu}
\caption{Alterações feitas no \textit{link} de menu}
\label{img:alteracao_menu}
\end{figure}

Para a segunda \textit{sprint} decidiu-se elaborar as funcionalidades de acompanhar projeto e exibir métricas. Foram entrevistadas as mesmas pessoas da primeira Iteração e seguiu-se as mesmas atividades da Iteração anterior. Os resultados obtidos estão na Tabela \ref{tabela_iteracao2}. A partir do gráfico presente na Figura \ref{img:grafico_iteracao2} percebe-se que o Usuário 1 apresentou um comportamento fora do esperado, isso se deve ao fato de que o examinado havia se confundido quanto ao enunciado da Tarefa 5.


\begin{table}[h!]
\centering
\caption{Tempo de Execução de Cada Atividade em Segundos 2a Iteração}
\label{tabela_iteracao2}
\begin{tabular}{|llllllll|}
\hline
\multicolumn{8}{|c|}{\cellcolor[HTML]{C0C0C0}\textbf{2a Iteração}}                                                   \\ \hline
                   & \textbf{T1} & \textbf{T2} & \textbf{T3} & \textbf{T4} & \textbf{T5} & \textbf{T6} & \textbf{T7} \\ \hline
\textbf{Usuário 1} & 0.04        & 0.32        & 0.11        & 0.19        & 0.37        & 0.07        & 0.07        \\ \hline
\textbf{Usuário 2} & 0.05        & 0.2         & 0.07        & 0.16        & 0.07        & 0.32        & 0.05        \\ \hline
\textbf{Usuário 3} & 0.05        & 0.17        & 0.05        & 0.14        & 0.08        & 0.27        & 0.02        \\ \hline
\textbf{Usuário 4} & 0.04        & 0.2         & 0.02        & 0.13        & 0.08        & 0.3         & 0.04        \\ \hline
\textbf{Usuário 5} & 0.05        & 0.22        & 0.12        & 0.15        & 0.07        & 0.35        & 0.05        \\ \hline
\textbf{Usuário 6} & 0.05        & 0.24        & 0.16        & 0.16        & 0.07        & 0.26        & 0.05        \\ \hline
\textbf{Usuário 7} & 0.06        & 0.2         & 0.16        & 0.18        & 0.07        & 0.05        & 0.06        \\ \hline
\end{tabular}
\end{table}

\graphicspath{{figuras/}}
\begin{figure}[h!]
\centering
\includegraphics[scale=0.75]{grafico_2a_iteracao}
\caption{Gráfico Comparativo das Atividades Realizadas na Iteração 2}
\label{img:grafico_iteracao2}
\end{figure}

 Uma das mudanças implementadas nessa segunda iteração se deve ao acréscimo de um botão para adicionar projeto na própria pagina inicial da aplicação. Este pedido havia sido feito na 1a Iteração por um dos entrevistados. Está alteração pode ser vista na Figura \ref{img:compara_botao}.
 
 
 \graphicspath{{figuras/}}
 \begin{figure}[h!]
 \centering
 \includegraphics[scale=0.60]{compara_adicao_botao}
 \caption{Acréscimo do Botão de Adicionar Projeto na Página Inicial}
 \label{img:compara_botao}
 \end{figure}
 
 
 
 
\begin{table}[h!]
\centering
\caption{\textit{Status} Completude do Trabalho}
\label{table_status}
\begin{tabular}{lc}
\rowcolor[HTML]{9A0000} 
{\color[HTML]{FFFFFF} \textbf{Atividade}} & \multicolumn{1}{l}{\cellcolor[HTML]{9A0000}{\color[HTML]{FFFFFF} \textbf{Status}}} \\ \hline
Definir Tema                              & 100\%                                                                              \\ \hline
Validar Escopo                            & 100\%                                                                              \\ \hline
Elaborar Roteiro de Pesquisa              & 100\%                                                                              \\ \hline
Pesquisar Referência                      & 100\%                                                                              \\ \hline
Refinar Pesquisa                          & 100\%                                                                              \\ \hline
Catalogar Material                        & 100\%                                                                              \\ \hline
Documentar                                & 100\%                                                                               \\ \hline
Analisar Ambiente                         & 100\%                                                                               \\ \hline
Configurar Ambiente                       & 100\%                                                                                \\ \hline
Implementar Solução de Software           & 100\%                                                                                \\ \hline
Implantar Solução em Ambiente Simulado    & 100\%                                                                                \\ \hline
Acompanhar Utilização do Software         & 0\%                                                                                \\ \hline
Aplicar Questionário                      & 0\%                                                                                \\ \hline
\end{tabular}
\end{table}
