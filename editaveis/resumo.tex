\begin{resumo}
Tão dificil quanto criar um software de qualidade é acompanhar a qualidade deste software. O tempo gasto com manutenção de software supera em muito o tempo de produção do software. Visando uma solução para este problema, este trabalho tem como objetivo propor uma solução de software voltado à visualização de métricas específicas para dois projetos escolhidos como sendo base para a utilização em um orgão público federal brasileiro. O artigo utilizou de revisão sistemática para obter o maior numero de informações pertinentes ao assunto nas principais bases pesquisas cientificas.

 \vspace{\onelineskip}
 
    
 \noindent
 \textbf{Palavras-chaves}: qualidade. dashboard. monitoramento de métricas. visualização de métricas.
\end{resumo}
