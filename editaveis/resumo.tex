\begin{resumo}
A contratação de software é uma prática comum dentro dos Órgão Públicos Brasileiros. No momento que a empresa ganha a licitação de um serviço, a empresa passa a ter o direito de contratação, para prestar o serviço ao Órgão contratante. Este trabalho tem por objetivo implementar um \textit{dashboard} que, com a ajuda de recursos visuais, seja uma ferramenta que auxilie no processo de contratação de software em Órgãos Públicos. Para esta solução, o \textit{dashboard} funciona como uma ferramenta de visualização de indicadores, determinados pelo Gestor de Projetos do Órgão Público. O \textit{dashboard} exibe os indicadores dos projetos, que estão sendo desenvolvidos pela Contratante, para o Gestor. O trabalho consistiu de duas fases: Iniciação e Execução. O objetivo da primeira fase é, estabelecer fundamentos (teóricos e arquiteturais) para a fase de Execução, enquanto o foco da segunda fase está na implementação da solução. Optou-se pelo uso, de uma adaptação da metodologia Scrum para o desenvolvimento da solução, contudo algumas mudanças foram necessárias, principalmente devido ao escopo e ao número de papéis, que a metodologia exige. Utilizou-se de Aprendizado de Máquina na sugestão de possíveis métricas para os gestores, baseado no perfis de outros gestores. Neste caso criou-se personas para representar estes perfis.
 \vspace{\onelineskip}
 
    
 \noindent
 \textbf{Palavras-chaves}: dashboard. qualidade. contratação de software. monitoramento de métricas. visualização de indicadores. aprendizado de maquina. personas.

\end{resumo}
