
\chapter[Suporte Tecnológico]{Suporte Tecnológico}

O objetivo desta seção é explicitar todo o aparato tecnológico a nível de software que foi utilizado durante o desenvolvimento deste trabalho.  Esta seção está dividida em \textit{Ferramentas para programação}.

\section{Ferramentas para Programação} % (fold)
\label{sec:engenharia_de_software}
	Neste tópico, serão apresentadas ferramentas e tecnologias voltadas ao contexto da Engenharia de Software que são utilizadas durante este trabalho, como, por exemplo, ferramentas para gerência de configuração e versionamento dos artefatos gerados.

	\subsection{GIT} % (fold)
	\label{sub:git}
	
		A ferramenta GIT\footnote{https://git-scm.com/} foi desenvolvida por Linus Torvalds durante a criação do Kernel Linux, pois Linus percebeu que existia a necessidade de criar uma ferramenta \textit{open-source} que fizesse o controle de versão \cite{bento_alise_2013}. 
		\\	O motivo para escolha da ferramenta se deve ao fato de que o Git contém o suporte para desenvolvimento linear o que garante um paralelismo de diversas áreas do desenvolvimento. Outro diferencial do Git está nos \textit{snapshots} dos objetos que são armazenados, isso significa que o Git não rearmazena arquivos que não foram alterados \cite{martinho_git_2013}.
	% subsection git (end)

	\subsection{Github} % (fold)
	\label{sub:github}
		O Github\footnote{https://github.com} é um repositório \textit{online} que fornece a criação de projetos públicos gratuitos. A ferramenta também provê um sistema de gestão para acompanhamento do desenvolvimento envolvendo um sistema de \textit{logs}, gráficos de visualização e uma \textit{Wiki} integrada a cada projeto \cite{martinho_git_2013}.
	
	 O principal motivo pela escolha do Github é que deseja-se disponibilizar a solução futuramente para consulta e aprimoramento de pessoas interessadas. 		
	% subsection github (end)

	\subsection{Bonita} % (fold)
	\label{sub:Bonita}
		 A ferramenta Bonita\footnote{http://www.bonitasoft.com/} foi escolhida graças a sua facilidade de utilização e portabilidade para o sistema operacional Mac OS X e o fato de ser gratuita. Esta ferramenta auxilia na modelagem de processos.
	% subsection bizagi_process_modeler (end)

	\subsection{Mac OS X} % (fold)
	\label{sub:Mac OS X}
		O sistema operacional Mac OS foi baseado no kernel Unix e fabricado e desenvolvido pela empresa Apple Inc. Utilizou-se a versão 10.11 do sistema também conhecida como "\textit{El Capitan}".
	% subsection linux_mint (end)

	\subsection{LaTeX} % (fold)
	\label{sub:latex}
	
	O LaTeX\footnote{https://www.latex-project.org/} foi desenvolvido na década de 80 cujo objetivo era simplificar a diagramação de textos científicos e matemáticos, onde atualmente dispõe de uma grande quantidade de macros para bibliografia, referencias, gráficos entre outros.
	% subsection latex (end)

	\subsection{Sublime Text 3} % (fold)
	\label{sub:sublime_text_3}
		O Sublime Text 3\footnote{https://www.sublimetext.com/3} é um editor de texto bastante utilizado por programadores, por possuir apoio para diversas linguagens de programação, incluindo textos em LaTeX.
	% subsection sublime_text_3 (end)

	\subsection{Zotero} % (fold)
	\label{sub:zotero}
		Zotero \footnote{https://www.zotero.org} é um software para gerenciamento de referências bibliográficas. Ele possui integração com o \textit{browser}, sincronização online e criação de bibliografias estilizadas.
	
